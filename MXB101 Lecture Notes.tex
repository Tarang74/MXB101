%!TEX program = xelatex
\documentclass{article}
\usepackage{LaTeX-Submodule/template}

% Additional packages & macros
\usepackage{multicol}

% Header and footer
\newcommand{\unitName}{Probability and Stochastic Modelling 1}
\newcommand{\unitTime}{Semester 1, 2022}
\newcommand{\unitCoordinator}{Dr Alexander Browning}
\newcommand{\documentAuthors}{\textsc{Tarang Janawalkar}}

\fancyhead[L]{\unitName}
\fancyhead[R]{\leftmark}
\fancyfoot[C]{\thepage}

% Copyright
\usepackage[
    type={CC},
    modifier={by-nc-sa},
    version={4.0},
    imagewidth={5em},
    hyphenation={raggedright}
]{doclicense}

\date{}

\begin{document}
%
\begin{titlepage}
    \vspace*{\fill}
    \begin{center}
        \LARGE{\textbf{\unitName}} \\[0.1in]
        \normalsize{\unitTime} \\[0.2in]
        \normalsize\textit{\unitCoordinator} \\[0.2in]
        \documentAuthors
    \end{center}
    \vspace*{\fill}
    \doclicenseThis
    \thispagestyle{empty}
\end{titlepage}
\newpage
%
\tableofcontents
\newpage
%
\section{Events and Probability}
\begin{definition}[Event]
    An event is a set of outcomes of a random experiment commonly denoted by a capital letter.

    Events can be simple (a single event) or compound (two or more simple events).
\end{definition}
\begin{definition}[Sample Space]
    The set of all possible outcomes of an experiment is known as the sample space
    for that experiment and is denoted \(\Omega\).

    All events \(A\) are subsets of \(\Omega\), \(A \subseteq \Omega\).
\end{definition}
\begin{definition}[Intersection]
    An intersection between two events \(A\) and \(B\) describes the set of outcomes that occur in both \(A\) and \(B\).

    An intersection can be represented using the set \ttfamily{AND} operator \(A \cup B\).
\end{definition}
 
\end{document}
