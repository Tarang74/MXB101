%!TEX program = xelatex
\documentclass{article}
\usepackage{LaTeX-Submodule/template}

% Additional packages & macros
\usepackage{multicol}

% Header and footer
\newcommand{\unitName}{Probability and Stochastic Modelling 1}
\newcommand{\unitTime}{Semester 1, 2022}
\newcommand{\unitCoordinator}{Dr Alexander Browning}
\newcommand{\documentAuthors}{\textsc{Tarang Janawalkar}}

\fancyhead[L]{\unitName}
\fancyhead[R]{\leftmark}
\fancyfoot[C]{\thepage}

% Copyright
\usepackage[
    type={CC},
    modifier={by-nc-sa},
    version={4.0},
    imagewidth={5em},
    hyphenation={raggedright}
]{doclicense}

\date{}

\begin{document}
%
\begin{titlepage}
    \vspace*{\fill}
    \begin{center}
        \LARGE{\textbf{\unitName}} \\[0.1in]
        \normalsize{\unitTime} \\[0.2in]
        \normalsize\textit{\unitCoordinator} \\[0.2in]
        \documentAuthors
    \end{center}
    \vspace*{\fill}
    \doclicenseThis
    \thispagestyle{empty}
\end{titlepage}
\newpage
%
\tableofcontents
\newpage
%
\section{Events and Probability}
\subsection{Events}
\begin{definition}[Event]
    An event is a set of outcomes in a random experiment commonly denoted by a capital letter.
    Events can be simple (a single event) or compound (two or more simple events).
\end{definition}
\begin{definition}[Sample space]
    The set of all possible outcomes of an experiment is known as the sample space
    for that experiment and is denoted \(\Omega\).
\end{definition}
\begin{definition}[Intersection]
    An intersection between two events \(A\) and \(B\) describes the set of outcomes that occur in both \(A\) and \(B\).
    The intersection can be represented using the set {\ttfamily{AND}} operator (\(\cap\)) --- \(A \cap B\) (or \(AB\)).
\end{definition}
\begin{definition}[Disjoint]
    Disjoint (mutually exclusive) events are two events that cannot occur simultaneously, or have no common outcomes.
\end{definition}
\begin{theorem}[Intersection of disjoint events]
    The intersection of disjoint events results in the null set (\(\varnothing\)).
\end{theorem}
\begin{lemma}
    Disjoint events are \textbf{dependent} events as the occurrence of one means the other cannot occur.
\end{lemma}
\begin{definition}[Union]
    A union of two events \(A\) and \(B\) describes the set of outcomes in either \(A\) or \(B\).
    The union is represented using the set {\ttfamily{OR}} operator (\(\cup\)) --- \(A \cup B\).
\end{definition}
\begin{definition}[Complement]
    The complement of an event \(E\) is the set of all other outcomes in \(\Omega\).
    The complement of \(E\) is denoted \(\overline{E}\).
\end{definition}
\begin{theorem}[Intersection of complement set]
    \begin{equation*}
        A\overline{A} = \varnothing
    \end{equation*}
\end{theorem}
\begin{theorem}[Union of complement set]
    \begin{equation*}
        A \cup \overline{A} = \Omega
    \end{equation*}
\end{theorem}
\begin{definition}[Subset]
    \(A\) is a (non-strict) subset of \(B\) if all elements in \(A\) are also in \(B\).
    This can be denoted as \(A \subseteq B\).
\end{definition}
\begin{theorem}
    All events \(E\) are subsets of \(\Omega\).
\end{theorem}
\begin{theorem}
    Given \(A \subseteq B\)
    \begin{equation*}
        AB = A \quad\quad \text{and} \quad\quad A \cup B = B
    \end{equation*}
\end{theorem}
\begin{corollary}
    Given \(\varnothing \subseteq E\)
    \begin{equation*}
        \varnothing E = \varnothing \quad\quad \text{and} \quad\quad \varnothing \cup E = E
    \end{equation*}
\end{corollary}
\begin{theorem}[Associative Identities]
    \begin{align*}
        A \left( BC \right) & = \left( AB \right) C \\
        A \cup \left( B \cup C \right) & = \left( A \cup B \right) \cup C
    \end{align*}
\end{theorem}
\begin{theorem}[Distributive Identities]
    \begin{align*}
        A \left(B \cup C\right) & = AB \cup AC \\
        A \cup BC & = \left( A \cup B \right) \left( A \cup C \right)
    \end{align*}
\end{theorem}
\subsection{Probability}
\begin{definition}[Probability]
    Probability is a measure of the likeliness of an event occurring. The probability of
    an event \(E\) is denoted \(\Pr{\left( E \right)}\) (sometimes \(\mathrm{P}\left( E \right)\)).
    \begin{equation*}
        0 \le \Pr{\left( E \right)} \le 1
    \end{equation*}
    where a probability of 0 never happens, and 1 always happens.
\end{definition}
\begin{theorem}[Probability of \(\Omega\)]
    \begin{equation*}
        \Pr{\left( \Omega \right)} = 1
    \end{equation*}
\end{theorem}
\begin{theorem}[Multiplicative rule]
    The probability of the intersection between two independent events \(A\) and \(B\) is given by
    \begin{equation*}
        \Pr{\left( AB \right)} = \Pr{\left( A \right)} \Pr{\left( B \right)}
    \end{equation*}
\end{theorem}
\begin{theorem}[Probability of disjoint events]
    The probability of disjoint events \(A\) and \(B\) is given by
    \begin{align*}
        \Pr{\left( AB \right)}          & = 0  \\
        \Pr{\left( \varnothing \right)} & = 0.
    \end{align*}
    Disjoint events are dependent events, since the occurrence of one means the other cannot occur.
\end{theorem}
\begin{theorem}[Addition rule]
    The probability of the union between two independent events \(A\) and \(B\) is given by
    \begin{equation*}
        \Pr{\left( A \cup B \right)} = \Pr{\left( A \right)} + \Pr{\left( B \right)} - \Pr{\left( AB \right)}.
    \end{equation*}
    If \(A\) and \(B\) are disjoint, then \(\Pr{\left( AB \right)} = 0\), so that \(\Pr{\left( A \cup B \right)} = \Pr{\left( A \right)} + \Pr{\left( B \right)}\).
\end{theorem}
\begin{corollary}[Addition rule for 3 event]
    The addition rule for 3 events is as follows
    \begin{equation*}
        \Pr{\left( A \cup B \cup C \right)} = \Pr{\left( A \right)} + \Pr{\left( B \right)} + \Pr{\left( B \right)} - \Pr{\left( AB \right)} - \Pr{\left( AC \right)} - \Pr{\left( BC \right)} + \Pr{\left( ABC \right)}.
    \end{equation*}
\end{corollary}
\begin{proof}
    If we write \(D = A \cup B\) and apply the addition rule twice, we have
    \begin{align*}
        \Pr{\left( A \cup B \cup C \right)} & = \Pr{\left( D \cup C \right)} \\
                                            & = \Pr{\left( D \right)} + \Pr{\left( C \right)} - \Pr{\left( DC \right)} \\
                                            & = \Pr{\left( A \cup B \right)} + \Pr{\left( C \right)} - \Pr{\left( \left( A \cup B \right)C \right)} \\
                                            & = \Pr{\left( A \right)} + \Pr{\left( B \right)} - \Pr{\left( AB \right)} + \Pr{\left( C \right)} - \Pr{\left( AC \cup BC \right)} \\
                                            & = \Pr{\left( A \right)} + \Pr{\left( B \right)} - \Pr{\left( AB \right)} + \Pr{\left( C \right)} - \left( \Pr{\left( AC \right)} + \Pr{\left( BC \right)} - \Pr{\left( ACBC \right)} \right) \\
                                            & = \Pr{\left( A \right)} + \Pr{\left( B \right)} + \Pr{\left( C \right)} - \Pr{\left( AB \right)} - \Pr{\left( AC \right)} - \Pr{\left( BC \right)} + \Pr{\left( ABC \right)}
    \end{align*}
\end{proof}
\begin{theorem}[Complement rule]
    The probability of the complement of \(E\) is given by
    \begin{equation*}
        \Pr{\left( \overline{E} \right)} = 1 - \Pr{\left( E \right)}
    \end{equation*}
\end{theorem}
\begin{theorem}[Probability of subsets]
    If \(A \subseteq B\) then \(\Pr{\left( A \right)} \le \Pr{\left( B \right)}\).
    Also, \(\Pr{\left( AB \right)} = \Pr{\left( A \right)}\) and \(\Pr{\left( A \cup B \right)} = \Pr{\left( B \right)}\).
\end{theorem}
\begin{theorem}[Law of total probability]
    By writing the event \(A\) as \(AB \cup A\overline{B}\), and noting that \(AB\) and \(A\overline{B}\) are disjoint:
    \begin{equation*}
        \Pr{\left( A \right)} = \Pr{\left( AB \right)} + \Pr{\left( A\overline{B} \right)}
    \end{equation*}
\end{theorem}
\begin{theorem}[De Morgan's laws]
    Recall De Morgan's Laws:
    \begin{align*}
        \overline{A \cup B} & = \overline{A} \ \overline{B}     \\
        \overline{AB}       & = \overline{A} \cup \overline{B}.
    \end{align*}
    Taking the negation of both sides and applying the complement rule yields
    \begin{align*}
        \Pr{\left( A \cup B \right)} & = 1 - \Pr{\left( \overline{A} \ \overline{B} \right)}    \\
        \Pr{\left( AB \right)}       & = 1 - \Pr{\left( \overline{A} \cup \overline{B} \right)}
    \end{align*}
\end{theorem}
\subsection{Circuits}
A signal can pass through a circuit if there is a functional path from start to finish.

We can define a circuit where each component \(i\) functions with probability \(p\), 
and is independent of other components.

Then \(W_i\) to be the event in which the associated component \(i\) functions, we can 
determine the event \(S\) in which the system functions,
and probability \(\Pr{\left( S \right)}\) that the system functions.

As the probability that any component functions is \(p\), in other words
\begin{equation*}
    \Pr{\left( W_i \right)} = p,
\end{equation*}
\(\Pr{\left( S \right)}\) will be a function of \(p\) defined \(f:\left( 0,\; 1 \right) \to \left( 0,\; 1 \right)\).
\end{document}
